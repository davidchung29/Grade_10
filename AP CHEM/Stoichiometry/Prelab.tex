% Options for packages loaded elsewhere
\PassOptionsToPackage{unicode}{hyperref}
\PassOptionsToPackage{hyphens}{url}
%
\documentclass[
]{article}
\usepackage{lmodern}
\usepackage{amssymb,amsmath}
\usepackage{ifxetex,ifluatex}
\ifnum 0\ifxetex 1\fi\ifluatex 1\fi=0 % if pdftex
  \usepackage[T1]{fontenc}
  \usepackage[utf8]{inputenc}
  \usepackage{textcomp} % provide euro and other symbols
\else % if luatex or xetex
  \usepackage{unicode-math}
  \defaultfontfeatures{Scale=MatchLowercase}
  \defaultfontfeatures[\rmfamily]{Ligatures=TeX,Scale=1}
\fi
% Use upquote if available, for straight quotes in verbatim environments
\IfFileExists{upquote.sty}{\usepackage{upquote}}{}
\IfFileExists{microtype.sty}{% use microtype if available
  \usepackage[]{microtype}
  \UseMicrotypeSet[protrusion]{basicmath} % disable protrusion for tt fonts
}{}
\makeatletter
\@ifundefined{KOMAClassName}{% if non-KOMA class
  \IfFileExists{parskip.sty}{%
    \usepackage{parskip}
  }{% else
    \setlength{\parindent}{0pt}
    \setlength{\parskip}{6pt plus 2pt minus 1pt}}
}{% if KOMA class
  \KOMAoptions{parskip=half}}
\makeatother
\usepackage{xcolor}
\IfFileExists{xurl.sty}{\usepackage{xurl}}{} % add URL line breaks if available
\IfFileExists{bookmark.sty}{\usepackage{bookmark}}{\usepackage{hyperref}}
\hypersetup{
  hidelinks,
  pdfcreator={LaTeX via pandoc}}
\urlstyle{same} % disable monospaced font for URLs
\setlength{\emergencystretch}{3em} % prevent overfull lines
\providecommand{\tightlist}{%
  \setlength{\itemsep}{0pt}\setlength{\parskip}{0pt}}
\setcounter{secnumdepth}{-\maxdimen} % remove section numbering

\author{}
\date{}

\begin{document}

\hypertarget{hydrogen-peroxide-redox-titration-lab}{%
\section{Hydrogen peroxide Redox Titration
Lab}\label{hydrogen-peroxide-redox-titration-lab}}

\hypertarget{section}{%
\paragraph{09.27.2021}\label{section}}

\hypertarget{david-chung}{%
\paragraph{David Chung}\label{david-chung}}

\hypertarget{pre-lab}{%
\subsection{Pre Lab}\label{pre-lab}}

\hypertarget{purpose}{%
\subsubsection{Purpose}\label{purpose}}

To determine the percentage and molarity of \(H_2O_2\) in ordinary
over-the-counter solutions of hydrogen peroxide by tritating a solution
of \(0.02\) \(M\) \(KMnO_4\) into a measured amount of an unknown
concentration of hydrogen peroxide, \(H_2O_2.\)

\hypertarget{lab-variables}{%
\subsubsection{Lab Variables}\label{lab-variables}}

Independant - solution of \(0.02\) \(M\) \(KMnO_4\) Dependent -
concentration of \(H_2O_2\)

\hypertarget{lab-safety-considerations}{%
\subsubsection{Lab Safety
Considerations}\label{lab-safety-considerations}}

\(H_2O_2\)→ can cause fire, permanent eye damage, eye and skin
irritation or burns, severe digestive tract irritation , and blood
abnormalities \(KMnO_4\)→ can cause fire, severe eye and skin irritation
or burns, can cause respiratory tract irritation and possible burns, and
severe digestive tract irritation and possible burns \(H_2SO_4\)→ causes
eye and skin burns, digestive and respiratory tract burns, may be fatal
if mist is inhaled, can cause cancer, reacts violently with water and
other substances. May cause lung damage, absorbs moisture from air, and
corrosive to metal. \(H_2O\)→ non-hazardous

\hypertarget{materials}{%
\subsubsection{Materials}\label{materials}}

\begin{itemize}
\tightlist
\item
  1 burrets
\item
  Erlenmeyer flasks
\item
  Stir plate
\item
  Magnetic stir bar or self stir
\item
  Balance
\item
  Household hydrogen peroxide
\item
  \(0.02M\) potassium permanganate solution
\item
  \(3.0M\) sulfuric acid
\item
  \(0.1M\) manganese(II) sulfate
\item
  Distilled water
\end{itemize}

\hypertarget{procedures}{%
\subsubsection{Procedures}\label{procedures}}

\begin{enumerate}
\def\labelenumi{\arabic{enumi}.}
\tightlist
\item
  Obtain and wear goggles.
\item
  Rinse and fill a buret with standardized \(KMnO_4\) solution. Record
  the molarity of thesolution.
\item
  Determine the mass of a clean, dry Erlenmeyer flask to the correct
  number ofsignificant figures, and record.
\item
  From the buret on the front table, record the initial volume (again,
  to the correct number of significant figures). Add approximately
  \(1.5 mL\) of ordinary household hydrogen peroxide. Record the final
  buret volume.
\item
  Find the mass of the flask with the peroxide and record.
\item
  Add about \(35 mL\) of distilled water, \(5 mL\) of \(3.0\) \(M\)
  \(H2SO_4\), and \(3\) or \(4\) drops of 0.1 MMnSO4, which acts as a
  catalyst.
\item
  Record the volume in your \(KMnO_4\) buret, and titrate your sample to
  a pale pink end point
\item
  Calculate the percentage of \(H_2O_2\) in the original sample
\item
  Repeat titration if possibleat least tw to three times for consistent
  results.
\end{enumerate}

\hypertarget{pre-lab-questions}{%
\subsubsection{Pre-lab Questions}\label{pre-lab-questions}}

\(6\text{H}^{+} + 2\text{MnO}_4^{-} + 5\text{H}_2\text{O}_2 \rightarrow 2\text{Mn}^{+2} + 8\text{H}_2\text{O} + 5\text{O}_2\)
1. 5 electrons are transferred 2. Magnesium(\(\text{Mn}\)) is oxidized
while Oxygen(\(\text{O}\)) is reduced

\hypertarget{lab}{%
\subsection{Lab}\label{lab}}

\hypertarget{data-analysis-calculations}{%
\subsubsection{Data Analysis /
Calculations}\label{data-analysis-calculations}}

\begin{enumerate}
\def\labelenumi{\arabic{enumi}.}
\item
  \[0.01451\text{L} \times \frac{0.02\text{ mol MnO}_4^{-}}{1 \text{ L MnO}_4^{-}} = \boxed{2.90\times10^{-4}\text{ mol MnO}_4^{-}} \\ 0.0146\text{L} \times \frac{0.02\text{ mol MnO}_4^{-}}{1 \text{ L MnO}_4^{-}} = \boxed{2.92\times10^{-4}\text{ mol MnO}_4^{-}}\]
\item
  \[2.90\times10^{-4}\text{ mol MnO}_4^{-} \times \frac{5\text{ mol }}{2 \text{ mol}}=\boxed{7.25\times 10^{-4}\text{ mol H}_2\text{O}_2} \newline 2.92\times10^{-4}\text{ mol MnO}_4^{-} \times \frac{5\text{ mol }}{2 \text{ mol}}=\boxed{7.30\times 10^{-4}\text{ mol H}_2\text{O}_2}\]
\item
  \[\frac{7.25\times 10^{-4} \times 34.017 = 2.47 \times \cancel{10^{-2}g}}{2.47\times 10^{-2}g + (158.034 \times 2.90 \times{10^{-4}}g)= 7.05 \times \cancel{10^{-2}}}\times 100 = \boxed{35.04} \newline \frac{7.30\times 10^{-4} \times 34.017 = 2.48 \times \cancel{10^{-2}}g}{2.48 \times 10^{-2}g + (158.034 \times 2.92 \times{10^{-4}})g = 7.09 \times \cancel{10^{-2}}}\times100 = \boxed{34.98}\]
\item
  \[\frac{2}{2+5 = 7}\times 100 = \boxed{28.57}\]
\item
  \[ \frac{|0.3-0.3504|}{0.3} \times 100 = \boxed{16.80} \newline \frac{|0.3-0.3498|}{0.3} \times 100 = \boxed{16.60}\]
\end{enumerate}

\hypertarget{discussion}{%
\subsubsection{Discussion}\label{discussion}}

\begin{enumerate}
\def\labelenumi{\arabic{enumi}.}
\item
  Oxidation is losing electrons while reduction is gaining the electrons
  lost from the oxidation
\item
  \begin{itemize}
  \tightlist
  \item
    No change, since cleaning with water does not change the hydrogen
    perxide in a solution
  \item
    too low, since there might be other substances from previous uses in
    the buret or reservoir.
  \end{itemize}
\item
  The labeling of household hydrogen peroxide is pretty accurate since
  it had an average percent error of \(16.70\) percent
\item
  \begin{itemize}
  \tightlist
  \item
    \(2:5\)
  \item
    \(2.780 \times 10^-{4}\) moles
  \item
    \(6.95 \times 10^{-4}\)
  \item
    \(2.78 \times 10^{-2}\) M
  \item
    \(\frac{0.0439\text{g}}{25\text{g}} = \boxed{0.176}\)
  \end{itemize}
\end{enumerate}

\hypertarget{post-lab}{%
\subsection{Post Lab}\label{post-lab}}

\hypertarget{purpose-1}{%
\subsubsection{Purpose}\label{purpose-1}}

To determine the percentage and molarity of \(H_2O_2\) in ordinary
over-the-counter solutions of hydrogen peroxide by tritating a solution
of \(0.02\) \(M\) \(KMnO_4\) into a measured amount of an unknown
concentration of hydrogen peroxide, \(H_2O_2.\)

\hypertarget{abstract}{%
\subsubsection{Abstract}\label{abstract}}

This lab examined the concentration of Hydrogen Peroxide by titrating a
solution of \(0.02 \text{M KMnO}_4.\) In order to find the concentraion
of hydrogen peroxide, we titrated \(\text{M KMnO}_4\) until the solution
turned a pale pink. Then we measured the amoung of \(\text{M KMnO}_4\)
used to find the molar concentration of the hydrogen peroxide. Our
results showed that the molard concentraion of the hydrogen peroxide was
an average of \(0.29M,\) close to the \(0.3M\) actual value. The results
show that the claimed molar concentration of over-the-counter hydrogen
peroxide is fairly accurate.

\hypertarget{material-equipment}{%
\subsubsection{Material \& Equipment}\label{material-equipment}}

\begin{itemize}
\tightlist
\item
  1 burrets
\item
  Erlenmeyer flasks
\item
  Stir plate
\item
  Magnetic stir bar or self stir
\item
  Balance
\item
  Household hydrogen peroxide
\item
  0.02M0.02M potassium permanganate solution
\item
  3.0M3.0M sulfuric acid
\item
  0.1M0.1M manganese(II) sulfate
\item
  Distilled water
\end{itemize}

\hypertarget{procedure}{%
\subsubsection{Procedure}\label{procedure}}

\begin{enumerate}
\def\labelenumi{\arabic{enumi}.}
\tightlist
\item
  Get lab coat and goggles
\item
  Clean and fill a buret with the\(KMnO_4\) solution and record the
  molarity of the solution.
\item
  Find mass of clean flask.
\item
  Record the initial volume of the buret. Add approximately \(1.5 mL\)
  of ordinary hydrogen peroxide then record the final buret volume.
\item
  Find the mass of the flask with the peroxide and record.
\item
  Add about \(35 mL\) of distilled water, \(5 mL\) of \(3.0\) \(M\)
  \(H2SO_4\), and \(3\) or \(4\) drops of 0.1 MMnSO4, which acts as a
  catalyst.
\item
  Record the volume in your \(KMnO_4\) buret, and titrate your sample to
  a pale pink end point
\item
  Calculate the percentage of \(H_2O_2\) in the original sample
\item
  Repeat another time for a second result.
\end{enumerate}

\hypertarget{error-analysis}{%
\subsubsection{Error Analysis}\label{error-analysis}}

\[\text{Trial 1 } \frac{|0.3-0.3504|}{0.3} \times 100 = \boxed{16.80} \newline \text{Trial 2 }\frac{|0.3-0.3498|}{0.3} \times 100 = \boxed{16.60}\]

\end{document}
